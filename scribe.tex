\documentclass[11pt]{article}

\usepackage{fullpage,times}%charter}
\usepackage{color}

\usepackage{tikz}
\usetikzlibrary{arrows.meta}

%% macros
\newcommand{\ax}[1]{\texttt{AX}(#1)}
\newcommand{\ex}[1]{\texttt{EX}(#1)}
\newcommand{\af}[1]{\texttt{AF}(#1)}
\newcommand{\ef}[1]{\texttt{EF}(#1)}
\newcommand{\ag}[1]{\texttt{AG}(#1)}
\newcommand{\eg}[1]{\texttt{EG}(#1)}
\newcommand{\au}[2]{\texttt{A}(#1\ \texttt{U}\ #2)}
\newcommand{\eu}[2]{\texttt{E}(#1\ \texttt{U}\ #2)}
\newcommand{\sem}[1]{[\!\![#1]\!\!]}

\newcommand{\sol}[1]{{\color{blue}#1}}

\begin{document}

Disproved. Consider a Kripke structure with states $s_0, s_1, s_2$ and transitions $s_0 \rightarrow s_1 \rightarrow s_2$, where $s_0$ satisfies $\lnot p$, $s_1$ satisfies $p \land \lnot q$, and $s_2$ satisfies $q$. The path $\pi = s_0 \rightarrow s_1 \rightarrow s_2$ satisfies $\lu{\lf{p}}{q}$ since $s_0$ does not satisfy $q$ and $s_1$ satisfies $p$, but there is no state where $p$ holds true before the state where $q$ holds true.

Proved. $\ax{\af{p}}$ means that for every path starting from the current state, $p$ eventually holds true along the path. $\lf{\lx{p}}$ means that in all next states from the current state, $p$ holds true. Thus, if the current state satisfies $\ax{\af{p}}$, then in particular, $p$ eventually holds true along any path starting from the current state. Therefore, in any next state, $p$ must hold true since it eventually holds true along all paths, which satisfies $\lf{\lx{p}}$. Conversely, if the current state satisfies $\lf{\lx{p}}$, then in all next states, $p$ holds true, so in particular, $p$ eventually holds true along any path starting from the current state. Therefore, the current state satisfies $\ax{\af{p}}$.

Proved. To show that $\lf{\llg{p}} \Rightarrow \llg{\lf{p}}$ is equivalent to propositional constant true, we need to show that for any Kripke structure and any state, the formula is always satisfied. Suppose a Kripke structure has a state $s$ that does not satisfy $\lf{p}$, then by definition of $\lf{}$, there is some next state $s'$ such that $p$ does not hold true in $s'$. Then, since $s'$ is a successor of $s$, we have $\llg{p}$ is false in $s$. Thus, $\lf{\llg{p}}$ is true in $s$ since $\llg{p}$ is false in some future state of $s$, and $\llg{\lf{p}}$ is also true in $s$ since the globally true $p$ is also locally true in all future states of $s$. Therefore, $\lf{\llg{p}} \Rightarrow \llg{\lf{p}}$ is true in all states of the Kripke structure, which means the formula is equivalent to propositional constant true.







\end{document}