\documentclass[11pt]{article}

\usepackage{fullpage,times}%charter}
\usepackage{color}

\usepackage{tikz}
\usetikzlibrary{arrows.meta}

%% macros
\newcommand{\ax}[1]{\texttt{AX}(#1)}
\newcommand{\ex}[1]{\texttt{EX}(#1)}
\newcommand{\af}[1]{\texttt{AF}(#1)}
\newcommand{\ef}[1]{\texttt{EF}(#1)}
\newcommand{\ag}[1]{\texttt{AG}(#1)}
\newcommand{\eg}[1]{\texttt{EG}(#1)}
\newcommand{\au}[2]{\texttt{A}(#1\ \texttt{U}\ #2)}
\newcommand{\eu}[2]{\texttt{E}(#1\ \texttt{U}\ #2)}
\newcommand{\sem}[1]{[\!\![#1]\!\!]}

\newcommand{\sol}[1]{{\color{blue}#1}}

\begin{document}


a) $\ag{p}$: The formula $\ag{p}$ means that in all states reachable from the current state, $p$ must hold. In the given Kripke structure, $p$ holds in $s_0$ and $s_2$, so $s_0$ and $s_2$ satisfy $\ag{p}$.

b) $\af{p}$: The formula $\af{p}$ means that there exists a state reachable from the current state such that $p$ holds in that state. In the given Kripke structure, $p$ holds in $s_2$, so $s_1$ and $s_2$ satisfy $\af{p}$.

c) $\ag{\af{p}}$: The formula $\ag{\af{p}}$ means that for all states reachable from the current state, there exists a state reachable from that state such that $p$ holds in that state. In the given Kripke structure, $p$ holds in $s_2$, so only $s_2$ satisfies $\ag{\af{p}}$.

d) $\af{\ag{p}}$: The formula $\af{\ag{p}}$ means that there exists a state reachable from the current state such that in all states reachable from that state, $p$ holds. In the given Kripke structure, $p$ holds in $s_0$ and $s_2$, so only $s_0$ and $s_2$ satisfy $\af{\ag{p}}$.

a) $\ef{p}$: The formula $\ef{p}$ means that there exists an execution path starting from the current state such that $p$ holds in some state on that path.

b) $\eg{\neg p \land \ef{p}}$: The formula $\eg{\neg p \land \ef{p}}$ means that there exists a state reachable from the current state such that the property $\neg p \land \ef{p}$ holds in that state. The property $\neg p \land \ef{p}$ means that there exists an execution path starting from that state such that $p$ holds in some state on that path, and $\neg p$ holds in the current state.

a) Disprove: The property $\af{\ag{p}}$ only requires that there exists a state reachable from the current state such that in all states reachable from that state, $p$ holds. It does not guarantee that $p$ holds infinitely often along all paths from the current state.

b) Disprove: Similar to the previous part, the property $\af{\ag{p}}$ only requires that there exists a state reachable from the current state such that in all states reachable from that state, $p$ holds. It does not guarantee that $\neg p$ holds finitely many times along all paths from the current state.

c) Prove: The property $\eg{\ef{p}}$ means that there exists a state reachable from the current state such that there exists an execution path starting from that state such that $p$ holds in some state on that path. This implies that $p$ holds infinitely often along at least one path from the current state.

d) Disprove: The formula $\au{p}{\ax{q}}$ means that there exists a state reachable from the current state such that $p$ holds in that state, and



$AG((p \rightarrow AF(\neg p)) \wedge (\neg p \rightarrow AF(p)))$

This formula can be read as "For all paths, it is globally true that if $p$ is true, then eventually it is followed by a state where $p$ is false, and if $p$ is false, then eventually it is followed by a state where $p$ is true."

Here's an explanation of why this formula captures the desired property:

$p \rightarrow AF(\neg p)$ means "if $p$ is true, then eventually it becomes false." This ensures that along all paths, whenever $p$ is true, it is eventually followed by a state where $p$ is false.
$\neg p \rightarrow AF(p)$ means "if $p$ is false, then eventually it becomes true." This ensures that along all paths, whenever $p$ is false, it is eventually followed by a state where $p$ is true.
By using the $AG$ operator, we are checking that these properties hold globally for all paths in the model. Therefore, this formula captures the desired property.

\newpage 

Yes, the property "for all paths of the form $\pi$, $p$ is true in every $\pi[i]$ where $i$ is even" can be expressed in CTL (Computation Tree Logic).

We can break down the property into two parts:

All paths of the form $\pi$:
This part of the property can be expressed using the universal path quantifier $\forall$ as follows:
$\forall \pi$

This specifies that the property should hold for all possible paths.

$p$ is true in every $\pi[i]$ where $i$ is even:
This part of the property can be expressed using the path quantifier $A$ (for "for all") and the temporal operator $G$ (for "globally") as follows:
$A(G(p)) \land A(G(\neg p))$

This specifies that $p$ should be true in every even position, and false in every odd position.

Putting it all together, we get:

$\forall \pi . A(G(p)) \land A(G(\neg p))$

This expresses the property "for all paths of the form $\pi$, $p$ is true in every $\pi[i]$ where $i$ is even" in CTL.



2,b 


The property described in the question is known as "eventually alternating." In other words, along any path in the system, the truth value of the proposition $p$ alternates between true and false, and eventually it reaches a state where it stays false or a state where it stays true.

This property can be expressed using temporal logic as follows:

$$\textbf{AG}(\textbf{AF}(p \land \textbf{AX}\neg p) \land \textbf{AF}(\neg p \land \textbf{AX}p))$$

where $\textbf{AF}$ means "eventually in the future" and $\textbf{AG}$ means "always in the future." The formula states that it is always true that, in the future, $p$ will eventually hold and then eventually not hold, and that $\neg p$ will eventually hold and then eventually hold again.

Intuitively, this means that the system keeps oscillating between states where $p$ is true and states where $p$ is false, and this oscillation happens infinitely often along every path in the system.

\end{document}