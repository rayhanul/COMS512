\documentclass[11pt]{article}

\usepackage{fullpage,times}%charter}
\usepackage{color}

\usepackage{tikz}
\usetikzlibrary{arrows.meta}

%% macros
\newcommand{\ax}[1]{\texttt{AX}(#1)}
\newcommand{\ex}[1]{\texttt{EX}(#1)}
\newcommand{\af}[1]{\texttt{AF}(#1)}
\newcommand{\ef}[1]{\texttt{EF}(#1)}
\newcommand{\ag}[1]{\texttt{AG}(#1)}
\newcommand{\eg}[1]{\texttt{EG}(#1)}
\newcommand{\au}[2]{\texttt{A}(#1\ \texttt{U}\ #2)}
\newcommand{\eu}[2]{\texttt{E}(#1\ \texttt{U}\ #2)}
\newcommand{\sem}[1]{[\!\![#1]\!\!]}

\newcommand{\sol}[1]{{\color{blue}#1}}

\begin{document}

Let's first expand the two LTL formulas:

$\lu{(\neg q)}{(\neg p \land \neg q)}$ means that at some point in the future, $\neg q$ holds and until that point, both $\neg p$ and $\neg q$ hold.
$\neg \lu{p}{q}$ means that it is not the case that $p$ holds until $q$ holds.
Now, let's assume that a state satisfies $\lu{(\neg q)}{(\neg p \land \neg q)}$. This means that there exists a future time $t$ at which $\neg q$ holds and until that time, both $\neg p$ and $\neg q$ hold.

Suppose for the sake of contradiction that the state also satisfies $\lu{p}{q}$, which means that $p$ holds until $q$ holds. Since $\neg p$ holds until the time $t$ and $p$ holds until $q$ holds, we have that $\neg q$ holds until $q$ holds. But this contradicts the fact that at time $t$, $\neg q$ holds. Therefore, the state cannot satisfy $\lu{p}{q}$.

In conclusion, if a state satisfies $\lu{(\neg q)}{(\neg p \land \neg q)}$, then it satisfies $\neg \lu{p}{q}$.

\end{document}