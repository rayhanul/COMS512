\documentclass[11pt]{article}

\usepackage{fullpage,times}%charter}
\usepackage{color}

\usepackage{tikz}
\usetikzlibrary{arrows.meta}

%% macros
\newcommand{\ax}[1]{\texttt{AX}(#1)}
\newcommand{\ex}[1]{\texttt{EX}(#1)}
\newcommand{\af}[1]{\texttt{AF}(#1)}
\newcommand{\ef}[1]{\texttt{EF}(#1)}
\newcommand{\ag}[1]{\texttt{AG}(#1)}
\newcommand{\eg}[1]{\texttt{EG}(#1)}
\newcommand{\au}[2]{\texttt{A}(#1\ \texttt{U}\ #2)}
\newcommand{\eu}[2]{\texttt{E}(#1\ \texttt{U}\ #2)}
\newcommand{\sem}[1]{[\!\![#1]\!\!]}

\newcommand{\sol}[1]{{\color{blue}#1}}

\begin{document}


The expression $\mathtt{P}_{\geq 1}(\lf{\neg q})$ represents the probability of observing at least one event where the proposition $\lf{\neg q}$ is true. Here, $\lf{\neg q}$ denotes the negation of the proposition $q$.

To compute this probability, we would need additional information about the probability distribution of the events in question. Without this information, we cannot compute the probability.

Assuming we have a probability distribution, one possible method to compute this probability is to use the complement rule of probability, which states that:

$\mathtt{P}_{\geq 1}(A) = 1 - \mathtt{P}_0(A)$

where $A$ is an event, $\mathtt{P}_{\geq 1}(A)$ is the probability of observing at least one occurrence of $A$, and $\mathtt{P}_0(A)$ is the probability of observing no occurrences of $A$.

Using this rule and the fact that $\neg q$ is the negation of $q$, we can write:

$\mathtt{P}_{\geq 1}(\lf{\neg q}) = 1 - \mathtt{P}_0(\lf{\neg q})$

where $\mathtt{P}_0(\lf{\neg q})$ is the probability of observing no occurrence of $\lf{\neg q}$.
\end{document}